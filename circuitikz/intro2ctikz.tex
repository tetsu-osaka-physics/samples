\documentclass[a4paper, papersize, dvipdfmx, bold]{jsarticle}
\usepackage{tetsu_physics}
\usepackage{siunitx}
\usepackage{listings}
\lstset{
basicstyle={\ttfamily},
identifierstyle={\small},
commentstyle={\smallitshape},
keywordstyle={\small\bfseries},
ndkeywordstyle={\small},
stringstyle={\small\ttfamily},
frame={tb},
breaklines=true,
columns=[l]{fullflexible},
% numbers=left,
xrightmargin=0zw,
xleftmargin=1zw,
% numberstyle={\scriptsize},
% stepnumber=1,
% numbersep=1zw,
lineskip=-0.5ex
}
\begin{document}
\begin{center}
  {\Large CircuiTikZまとめ}\par
  最終更新:\makeatletter\the\year 年\two@digits\month 月\two@digits\day 日
\end{center}

\section{描画の基本}
\texttt{=}でラベル,\texttt{i}で電流,\texttt{v}で電圧.

\bigskip

\begin{minipage}{0.15\hsize}
  \begin{circuitikz}
    \draw (0, 0) to [R=$r$\,\si{[\micro\ohm]}, i=$i$, v=$v$] (2, 0);
  \end{circuitikz}
\end{minipage}
\begin{minipage}{0.8\hsize}
\begin{lstlisting}
\begin{circuitikz}
\draw (0, 0) to [R=$r$\,\si{[\micro\ohm]}, i=$i$, v=$v$] (2, 0);
\end{circuitikz}
\end{lstlisting}
\end{minipage}

\bigskip

\begin{minipage}{0.15\hsize}
  \begin{circuitikz}
    \draw (0, 0) to [R=\SI{1}{[M\ohm]}, i=$i$, v=$v$] (2, 0);
  \end{circuitikz}
\end{minipage}
\begin{minipage}{0.8\hsize}
  \begin{lstlisting}
    \begin{circuitikz}
      \draw (0, 0) to [R=\SI{1}{[M\ohm]}, i=$i$, v=$v$] (2, 0);
    \end{circuitikz}
  \end{lstlisting}
\end{minipage}

\subsection{ラベル}
ラベルを上に出す場合は\texttt{l\textasciicircum},下に出す場合は\texttt{l\_}を使う.

\bigskip

\begin{minipage}{0.2\hsize}
  \begin{circuitikz}
    \draw (0, 0) to [R, l^=$R$] (2, 0) node [right] {\texttt{l\textasciicircum}};
  \end{circuitikz}
\end{minipage}
\begin{minipage}{0.75\hsize}
  \begin{lstlisting}
    \begin{circuitikz}
      \draw (0, 0) to [R, l^=$R$] (2, 0);
    \end{circuitikz}
  \end{lstlisting}
\end{minipage}

\bigskip

\begin{minipage}{0.2\hsize}
  \begin{circuitikz}
    \draw (0, 0) to [R, l_=$R$] (2, 0) node [right] {\texttt{l\_}};
  \end{circuitikz}
\end{minipage}
\begin{minipage}{0.75\hsize}
  \begin{lstlisting}
    \begin{circuitikz}
      \draw (0, 0) to [R, l_=$R$] (2, 0);
    \end{circuitikz}
  \end{lstlisting}
\end{minipage}

ラベルを上下に出したければ\texttt{a\textasciicircum}, \texttt{l\_}を使う.

\bigskip

\begin{minipage}{0.2\hsize}
  \begin{circuitikz}
    \draw (0, 0) to [R, a^=$R_1$, l_=$R_2$] (2, 0);
  \end{circuitikz}
\end{minipage}
\begin{minipage}{0.75\hsize}
  \begin{lstlisting}
    \begin{circuitikz}
      \draw (0, 0) to [R, a^=$R_1$, l_=$R_2$] (2, 0);
    \end{circuitikz}
  \end{lstlisting}
\end{minipage}

\subsection{電流}
電流は\texttt{i}の代わりに\texttt{f}も使える.
\verb|f=$i$|にすればいい感じに空気を読んでくれる.自分で場所を指定したい場合は以下の8通り.

\bigskip

\begin{minipage}{0.3\hsize}
  \begin{circuitikz}
    \draw (0, 0) to [R, l=$R$, f>^=$i$] (3, 0) node [right] {\texttt{f>\textasciicircum}};
  \end{circuitikz}
\end{minipage}
\begin{minipage}{0.65\hsize}
  \begin{lstlisting}
    \begin{circuitikz}
      \draw (0, 0) to [R, l=$R$, f>^=$i$] (3, 0);
    \end{circuitikz}
  \end{lstlisting}
\end{minipage}

\bigskip

\begin{minipage}{0.3\hsize}
  \begin{circuitikz}
    \draw (0, 0) to [R, l=$R$, f>_=$i$] (3, 0) node [right] {\texttt{f>\_}};
  \end{circuitikz}
\end{minipage}
\begin{minipage}{0.65\hsize}
  \begin{lstlisting}
    \begin{circuitikz}
      \draw (0, 0) to [R, l=$R$, f>_=$i$] (3, 0);
    \end{circuitikz}
  \end{lstlisting}
\end{minipage}

\bigskip

\begin{minipage}{0.3\hsize}
  \begin{circuitikz}
    \draw (0, 0) to [R, l=$R$, f^<=$i$] (3, 0) node [right] {\texttt{f\textasciicircum<}};
  \end{circuitikz}
\end{minipage}
\begin{minipage}{0.65\hsize}
  \begin{lstlisting}
    \begin{circuitikz}
      \draw (0, 0) to [R, l=$R$, f^<=$i$] (3, 0);
    \end{circuitikz}
  \end{lstlisting}
\end{minipage}

\bigskip

\begin{minipage}{0.3\hsize}
  \begin{circuitikz}
    \draw (0, 0) to [R, l=$R$, f_<=$i$] (3, 0) node [right] {\texttt{f\_<}};
  \end{circuitikz}
\end{minipage}
\begin{minipage}{0.65\hsize}
  \begin{lstlisting}
    \begin{circuitikz}
      \draw (0, 0) to [R, l=$R$, f_<=$i$] (3, 0);
    \end{circuitikz}
  \end{lstlisting}
\end{minipage}

\bigskip

\begin{minipage}{0.3\hsize}
  \begin{circuitikz}
    \draw (0, 0) to [R, l=$R$, f<^=$i$] (3, 0) node [right] {\texttt{f<\textasciicircum}};
  \end{circuitikz}
\end{minipage}
\begin{minipage}{0.65\hsize}
  \begin{lstlisting}
    \begin{circuitikz}
      \draw (0, 0) to [R, l=$R$, f<^=$i$] (3, 0);
    \end{circuitikz}
  \end{lstlisting}
\end{minipage}

\bigskip

\begin{minipage}{0.3\hsize}
  \begin{circuitikz}
    \draw (0, 0) to [R, l=$R$, f<_=$i$] (3, 0) node [right] {\texttt{f<\_}};
  \end{circuitikz}
\end{minipage}
\begin{minipage}{0.65\hsize}
  \begin{lstlisting}
    \begin{circuitikz}
      \draw (0, 0) to [R, l=$R$, f<_=$i$] (3, 0);
    \end{circuitikz}
  \end{lstlisting}
\end{minipage}

\bigskip

\begin{minipage}{0.3\hsize}
  \begin{circuitikz}
    \draw (0, 0) to [R, l=$R$, f^>=$i$] (3, 0) node [right] {\texttt{f\textasciicircum>}};
  \end{circuitikz}
\end{minipage}
\begin{minipage}{0.65\hsize}
  \begin{lstlisting}
    \begin{circuitikz}
      \draw (0, 0) to [R, l=$R$, f^>=$i$] (3, 0);
    \end{circuitikz}
  \end{lstlisting}
\end{minipage}

\bigskip

\begin{minipage}{0.3\hsize}
  \begin{circuitikz}
    \draw (0, 0) to [R, l=$R$, f_>=$i$] (3, 0) node [right] {\texttt{f\_>}};
  \end{circuitikz}
\end{minipage}
\begin{minipage}{0.65\hsize}
  \begin{lstlisting}
    \begin{circuitikz}
      \draw (0, 0) to [R, l=$R$, f_>=$i$] (3, 0);
    \end{circuitikz}
  \end{lstlisting}
\end{minipage}

\verb|xf=\relax|を使えば,電流の矢印に×がつく.

\bigskip

\begin{minipage}{0.3\hsize}
  \begin{circuitikz}
    \draw (0, 0) to [R, l=$R$, xf>^=\relax] (3, 0) node [right] {\texttt{xf>\textasciicircum}};
  \end{circuitikz}
\end{minipage}
\begin{minipage}{0.65\hsize}
  \begin{lstlisting}
    \begin{circuitikz}
      \draw (0, 0) to [R, l=$R$, xf>^=\relax] (3, 0);
    \end{circuitikz}
  \end{lstlisting}
\end{minipage}

\bigskip

\begin{minipage}{0.3\hsize}
  \begin{circuitikz}
    \draw (0, 0) to [R, l=$R$, xf>_=\relax] (3, 0) node [right] {\texttt{xf>\_}};
  \end{circuitikz}
\end{minipage}
\begin{minipage}{0.65\hsize}
  \begin{lstlisting}
    \begin{circuitikz}
      \draw (0, 0) to [R, l=$R$, xf>_=\relax] (3, 0);
    \end{circuitikz}
  \end{lstlisting}
\end{minipage}

\bigskip

\begin{minipage}{0.3\hsize}
  \begin{circuitikz}
    \draw (0, 0) to [R, l=$R$, xf^<=\relax] (3, 0) node [right] {\texttt{xf\textasciicircum<}};
  \end{circuitikz}
\end{minipage}
\begin{minipage}{0.65\hsize}
  \begin{lstlisting}
    \begin{circuitikz}
      \draw (0, 0) to [R, l=$R$, xf^<=\relax] (3, 0);
    \end{circuitikz}
  \end{lstlisting}
\end{minipage}

\bigskip

\begin{minipage}{0.3\hsize}
  \begin{circuitikz}
    \draw (0, 0) to [R, l=$R$, xf_<=\relax] (3, 0) node [right] {\texttt{xf\_<}};
  \end{circuitikz}
\end{minipage}
\begin{minipage}{0.65\hsize}
  \begin{lstlisting}
    \begin{circuitikz}
      \draw (0, 0) to [R, l=$R$, xf_<=\relax] (3, 0);
    \end{circuitikz}
  \end{lstlisting}
\end{minipage}

\bigskip

\begin{minipage}{0.3\hsize}
  \begin{circuitikz}
    \draw (0, 0) to [R, l=$R$, xf<^=\relax] (3, 0) node [right] {\texttt{xf<\textasciicircum}};
  \end{circuitikz}
\end{minipage}
\begin{minipage}{0.65\hsize}
  \begin{lstlisting}
    \begin{circuitikz}
      \draw (0, 0) to [R, l=$R$, xf<^=\relax] (3, 0);
    \end{circuitikz}
  \end{lstlisting}
\end{minipage}

\bigskip

\begin{minipage}{0.3\hsize}
  \begin{circuitikz}
    \draw (0, 0) to [R, l=$R$, xf<_=\relax] (3, 0) node [right] {\texttt{xf<\_}};
  \end{circuitikz}
\end{minipage}
\begin{minipage}{0.65\hsize}
  \begin{lstlisting}
    \begin{circuitikz}
      \draw (0, 0) to [R, l=$R$, xf<_=\relax] (3, 0);
    \end{circuitikz}
  \end{lstlisting}
\end{minipage}

\bigskip

\begin{minipage}{0.3\hsize}
  \begin{circuitikz}
    \draw (0, 0) to [R, l=$R$, xf^>=\relax] (3, 0) node [right] {\texttt{xf\textasciicircum>}};
  \end{circuitikz}
\end{minipage}
\begin{minipage}{0.65\hsize}
  \begin{lstlisting}
    \begin{circuitikz}
      \draw (0, 0) to [R, l=$R$, xf^>=\relax] (3, 0);
    \end{circuitikz}
  \end{lstlisting}
\end{minipage}

\bigskip

\begin{minipage}{0.3\hsize}
  \begin{circuitikz}
    \draw (0, 0) to [R, l=$R$, xf_>=\relax] (3, 0) node [right] {\texttt{xf\_>}};
  \end{circuitikz}
\end{minipage}
\begin{minipage}{0.65\hsize}
  \begin{lstlisting}
    \begin{circuitikz}
      \draw (0, 0) to [R, l=$R$, xf_>=\relax] (3, 0);
    \end{circuitikz}
  \end{lstlisting}
\end{minipage}

\subsection{電圧}
例の三角形の奴はそのうち作るかも.

\bigskip

\begin{minipage}{0.3\hsize}
  \begin{circuitikz}
    \draw (0, 0) to [R, l_=$R_1$, v^>=$v$] (3, 0) node [right] {\texttt{v\textasciicircum>}};
  \end{circuitikz}
\end{minipage}
\begin{minipage}{0.65\hsize}
  \begin{lstlisting}
    \begin{circuitikz}
      \draw (0, 0) to [R, l_=$R_1$, v^>=$v$] (3, 0);
    \end{circuitikz}
  \end{lstlisting}
\end{minipage}

\bigskip

\begin{minipage}{0.3\hsize}
  \begin{circuitikz}
    \draw (0, 0) to [R, l=$R_1$, v_>=$v$] (3, 0) node [right] {\texttt{v\_>}};
  \end{circuitikz}
\end{minipage}
\begin{minipage}{0.65\hsize}
  \begin{lstlisting}
    \begin{circuitikz}
      \draw (0, 0) to [R, l=$R_1$, v_>=$v$] (3, 0);
    \end{circuitikz}
  \end{lstlisting}
\end{minipage}

\bigskip

\begin{minipage}{0.3\hsize}
  \begin{circuitikz}
    \draw (0, 0) to [R, l_=$R_1$, v^<=$v$] (3, 0) node [right] {\texttt{v\textasciicircum<}};
  \end{circuitikz}
\end{minipage}
\begin{minipage}{0.65\hsize}
  \begin{lstlisting}
    \begin{circuitikz}
      \draw (0, 0) to [R, l_=$R_1$, v^<=$v$] (3, 0);
    \end{circuitikz}
  \end{lstlisting}
\end{minipage}

\bigskip

\begin{minipage}{0.3\hsize}
  \begin{circuitikz}
    \draw (0, 0) to [R, l=$R_1$, v_<=$v$] (3, 0) node [right] {\texttt{v\_<}};
  \end{circuitikz}
\end{minipage}
\begin{minipage}{0.65\hsize}
  \begin{lstlisting}
    \begin{circuitikz}
      \draw (0, 0) to [R, l=$R_1$, v_<=$v$] (3, 0);
    \end{circuitikz}
  \end{lstlisting}
\end{minipage}

\subsection{拡大縮小}
\texttt{scale}を指定する.素子の大きさも変更する場合は\texttt{transform shape}を付ける.

\bigskip

\begin{minipage}{0.3\hsize}
  \begin{circuitikz}[scale=0.75]
    \draw (0, 0) to [R, l=$R$, i=$i$, v=$v$] (3, 0);
  \end{circuitikz}
\end{minipage}
\begin{minipage}{0.65\hsize}
  \begin{lstlisting}
    \begin{circuitikz}[scale=0.75]
      \draw (0, 0) to [R, l=$R$, i=$i$, v=$v$] (2, 0);
    \end{circuitikz}
  \end{lstlisting}
\end{minipage}

\bigskip

\begin{minipage}{0.3\hsize}
  \begin{circuitikz}[scale=0.75, transform shape]
    \draw (0, 0) to [R, l=$R$, i=$i$, v=$v$] (3, 0);
  \end{circuitikz}
\end{minipage}
\begin{minipage}{0.65\hsize}
  \begin{lstlisting}
    \begin{circuitikz}[scale=0.75, transform shape]
      \draw (0, 0) to [R, l=$R$, i=$i$, v=$v$] (2, 0);
    \end{circuitikz}
  \end{lstlisting}
\end{minipage}

\subsection{端点処理}
\texttt{*}, \texttt{o}, \texttt{d}を使う.導線のみ描く場合は\texttt{short}を使う.

\bigskip

\begin{minipage}{0.3\hsize}
  \begin{circuitikz}
    \draw (0, 0) node [left] {\texttt{*}} to [R, *-o] (2, 0) node [right] {\texttt{o}};
  \end{circuitikz}
\end{minipage}
\begin{minipage}{0.65\hsize}
  \begin{lstlisting}
    \begin{circuitikz}
      \draw (0, 0) to [R, *-o] (2, 0);
    \end{circuitikz}
  \end{lstlisting}
\end{minipage}

\bigskip

\begin{minipage}{0.3\hsize}
  \begin{circuitikz}
    \draw (0, 0) node [left] {\texttt{d}} to [short, d-] (2, 0);
  \end{circuitikz}
\end{minipage}
\begin{minipage}{0.65\hsize}
  \begin{lstlisting}
    \begin{circuitikz}
      \draw (0, 0) to [short, d-] (2, 0);
    \end{circuitikz}
  \end{lstlisting}
\end{minipage}

\subsection{アンカー}
\texttt{name}で素子に名前を付けると,素子近傍の座標が定義される.

\begin{center}
  \begin{circuitikz}
    \draw (0, 0) to [R, name=Resistor] (2, 0);
    \draw[thick, <-, >=stealth, blue] (Resistor.left) to [out=135, in=270] +(-0.5, 1) node [above] {left};
    \draw[thick, <-, >=stealth, blue] (Resistor.center) -- +(0, 1) node [above] {center};
    \draw[thick, <-, >=stealth, blue] (Resistor.right) to [out=45, in=270] +(0.5, 1) node [above] {right};
    \begin{scope}[shift={(6, 0)}]
      \draw (0, 0) to [R, name=Resistor] (2, 0);
      \draw[thick, <-, >=stealth, blue] (Resistor.north) -- +(0, 1) node [above] {north};
      \draw[thick, <-, >=stealth, blue] (Resistor.north east) -- +(0.7, 0.5) node [above right] {north east};
      \draw[thick, <-, >=stealth, blue] (Resistor.east) -- +(1, 0) node [right] {east};
      \draw[thick, <-, >=stealth, blue] (Resistor.south east) -- +(0.7, -0.5) node [below right] {south east};
      \draw[thick, <-, >=stealth, blue] (Resistor.south) -- +(0, -1) node [below] {south};
      \draw[thick, <-, >=stealth, blue] (Resistor.south west) -- +(-0.7, -0.5) node [below left] {south west};
      \draw[thick, <-, >=stealth, blue] (Resistor.west) -- +(-1, 0) node [left] {west};
      \draw[thick, <-, >=stealth, blue] (Resistor.north west) -- +(-0.7, 0.5) node [above left] {north west};
    \end{scope}
  \end{circuitikz}
\end{center}

\bigskip

\begin{minipage}{0.25\hsize}
  \begin{circuitikz}
    \draw (0, 0) to [R, name=Resistor] (2, 0);
    \draw[thick, <-, >=stealth, blue] (Resistor.center) -- +(1, 1) node [above] {center};
  \end{circuitikz}
\end{minipage}
\begin{minipage}{0.7\hsize}
  \begin{lstlisting}
    \begin{circuitikz}
      \draw (0, 0) to [R, name=Resistor] (2, 0);
      \draw[thick, <-, >=stealth, blue] (Resistor.center) -- +(1, 1) node [above] {center};
    \end{circuitikz}
  \end{lstlisting}
\end{minipage}

\section{素子}
\subsection{接地}
\texttt{node}のオプションに指定する.

\bigskip

\begin{minipage}{0.3\hsize}
  \begin{circuitikz}
    \draw (0, 0) node [ground] {};
    \draw (1.2, -0.25) node {\texttt{ground}};
  \end{circuitikz}
\end{minipage}
\begin{minipage}{0.65\hsize}
  \begin{lstlisting}
    \begin{circuitikz}
      \draw (0, 0) node [ground] {};
    \end{circuitikz}
  \end{lstlisting}
\end{minipage}

\bigskip

\begin{minipage}{0.3\hsize}
  \begin{circuitikz}
    \draw (0, 0) node [nground] {};
    \draw (1.2, -0.25) node {\texttt{nground}};
  \end{circuitikz}
\end{minipage}
\begin{minipage}{0.65\hsize}
  \begin{lstlisting}
    \begin{circuitikz}
      \draw (0, 0) node [nground] {};
    \end{circuitikz}
  \end{lstlisting}
\end{minipage}

\bigskip

\begin{minipage}{0.3\hsize}
  \begin{circuitikz}
    \draw (0, 0) node [pground] {};
    \draw (1.2, -0.25) node {\texttt{pground}};
  \end{circuitikz}
\end{minipage}
\begin{minipage}{0.65\hsize}
  \begin{lstlisting}
    \begin{circuitikz}
      \draw (0, 0) node [pground] {};
    \end{circuitikz}
  \end{lstlisting}
\end{minipage}

\paragraph{Tips}
\texttt{node}を回転させる場合は\texttt{rotate}を指定する.

\bigskip

\begin{minipage}{0.3\hsize}
  \begin{circuitikz}
    \draw (0, 0) node [ground, rotate=90] {};
    \draw (0, -0.4) node [right] {\texttt{ground, rotate=90}};
  \end{circuitikz}
\end{minipage}
\begin{minipage}{0.65\hsize}
  \begin{lstlisting}
    \begin{circuitikz}
      \draw (0, 0) node [ground, rotate=90] {};
    \end{circuitikz}
  \end{lstlisting}
\end{minipage}

\subsection{抵抗}

\begin{minipage}{0.3\hsize}
  \begin{circuitikz}
    \draw (0, 0) to [R] (2, 0) node [right] {\texttt{R}};
  \end{circuitikz}
\end{minipage}
\begin{minipage}{0.65\hsize}
  \begin{lstlisting}
    \begin{circuitikz}
      \draw (0, 0) to [R] (2, 0);
    \end{circuitikz}
  \end{lstlisting}
\end{minipage}

\bigskip

\begin{minipage}{0.3\hsize}
  \begin{circuitikz}
    \draw (0, 0) to [vR] (2, 0) node [right] {\texttt{vR}};
  \end{circuitikz}
\end{minipage}
\begin{minipage}{0.65\hsize}
  \begin{lstlisting}
    \begin{circuitikz}
      \draw (0, 0) to [vR] (2, 0);
    \end{circuitikz}
  \end{lstlisting}
\end{minipage}

\bigskip

\begin{minipage}{0.3\hsize}
  \begin{circuitikz}
    \draw (0, 0) to [R, fill=black] (2, 0) node [right] {\texttt{R, fill=black}};
  \end{circuitikz}
\end{minipage}
\begin{minipage}{0.65\hsize}
  \begin{lstlisting}
    \begin{circuitikz}
      \draw (0, 0) to [R, fill=black] (2, 0);
    \end{circuitikz}
  \end{lstlisting}
\end{minipage}

\bigskip

\begin{minipage}{0.3\hsize}
  \begin{circuitikz}
    \draw (0, 0) to [R, american resistor] (2, 0);
    \draw (0, -0.5) node [right] {\texttt{R, american resistor}};
  \end{circuitikz}
\end{minipage}
\begin{minipage}{0.65\hsize}
  \begin{lstlisting}
    \begin{circuitikz}
      \draw (0, 0) to [R, american resistor] (2, 0);
    \end{circuitikz}
  \end{lstlisting}
\end{minipage}

\paragraph{Tips}
素子を(銅線に関して)反転させる場合は\texttt{mirror}を使う.

\bigskip

\begin{minipage}{0.3\hsize}
  \begin{circuitikz}
    \draw (0, 0) to [vR, mirror] (2, 0) node [right] {\texttt{vR, mirror}};
  \end{circuitikz}
\end{minipage}
\begin{minipage}{0.65\hsize}
  \begin{lstlisting}
    \begin{circuitikz}
      \draw (0, 0) to [vR, mirror] (2, 0);
    \end{circuitikz}
  \end{lstlisting}
\end{minipage}

\subsection{コンデンサー}
\begin{minipage}{0.35\hsize}
  \begin{circuitikz}
    \draw (0, 0) to [capacitor] (1.5, 0) node [right] {\texttt{capacitor}};
  \end{circuitikz}
\end{minipage}
\begin{minipage}{0.6\hsize}
  \begin{lstlisting}
    \begin{circuitikz}
      \draw (0, 0) to [capacitor] (1.5, 0);
    \end{circuitikz}
  \end{lstlisting}
\end{minipage}

\bigskip

\begin{minipage}{0.35\hsize}
  \begin{circuitikz}
    \draw (0, 0) to [wavy capacitor] (1.5, 0) node [right] {\texttt{wavy capacitor}};
  \end{circuitikz}
\end{minipage}
\begin{minipage}{0.6\hsize}
  \begin{lstlisting}
    \begin{circuitikz}
      \draw (0, 0) to [wavy capacitor] (1.5, 0);
    \end{circuitikz}
  \end{lstlisting}
\end{minipage}

\bigskip

\begin{minipage}{0.35\hsize}
  \begin{circuitikz}
    \draw (0, 0) to [variable capacitor] (1.5, 0) node [right] {\texttt{variable capacitor}};
  \end{circuitikz}
\end{minipage}
\begin{minipage}{0.6\hsize}
  \begin{lstlisting}
    \begin{circuitikz}
      \draw (0, 0) to [variable capacitor] (1.5, 0);
    \end{circuitikz}
  \end{lstlisting}
\end{minipage}

\bigskip

\begin{minipage}{0.35\hsize}
  \begin{circuitikz}
    \draw (0, 0) to [capacitor, \Cwidth=0.4] (1.5, 0);
    \draw (0, -1) node [right] {\texttt{capacitor, \textbackslash Cwidth=0.4}};
  \end{circuitikz}
\end{minipage}
\begin{minipage}{0.6\hsize}
  \begin{lstlisting}
    \begin{circuitikz}
      \draw (0, 0) to [capacitor, \Cwidth=0.4] (1.5, 0);
    \end{circuitikz}
  \end{lstlisting}
\end{minipage}

\paragraph{Tips}
コンデンサーの電荷等を描く場合は次のようにする.

\bigskip

\begin{minipage}{0.2\hsize}
  \begin{circuitikz}
    \draw (0, 0) to [capacitor, \Cwidth=0.4, name=capa] (2, 0);
    \draw (capa.center) node {$C$};
    \draw (capa.north east) node [right] {$+Q$};
    \draw (capa.north west) node [left] {$-Q$};
  \end{circuitikz}
\end{minipage}
\begin{minipage}{0.75\hsize}
  \begin{lstlisting}
    \begin{circuitikz}
      \draw (0, 0) to [capacitor, \Cwidth=0.4, name=capa] (2, 0);
      \draw (capa.center) node {$C$};
      \draw (capa.north east) node [right] {$+Q$};
      \draw (capa.north west) node [left] {$-Q$};
    \end{circuitikz}
  \end{lstlisting}
\end{minipage}

容量だけの場合は,
\begin{center}
  \texttt{\textbackslash drawCapa*}\{始点\}\{終点\}\{容量\}
\end{center}

\bigskip

\begin{minipage}{0.2\hsize}
  \begin{circuitikz}
    \drawCapa*{(0, 0)}{(2, 0)}{$C_1$};
  \end{circuitikz}
\end{minipage}
\begin{minipage}{0.75\hsize}
  \begin{lstlisting}
    \begin{circuitikz}
      \drawCapa*{(0, 0)}{(2, 0)}{$C_1$};
    \end{circuitikz}
  \end{lstlisting}
\end{minipage}

コンデンサーの容量・電荷を両方描く場合は,
\begin{center}
  \texttt{\textbackslash drawCapa[!]}\{始点\}\{終点\}\{容量\}\{電荷\}
\end{center}
銅線が横の場合に,右側の極板を$+$にするには\texttt{[r]}(省略可),左側の極板を$+$にするには\texttt{[l]}.
銅線が縦の場合に,上側の極板を$+$にするには\texttt{[u]},下側の極板を$+$にするには\texttt{[d]}.
\textbf{終点は終点より右・上に配置すること!}

\bigskip

\begin{minipage}{0.2\hsize}
  \begin{circuitikz}
    \drawCapa{(0, 0)}{(2, 0)}{$C_1$}{$Q_1$};
  \end{circuitikz}
\end{minipage}
\begin{minipage}{0.75\hsize}
  \begin{lstlisting}
    \begin{circuitikz}
      \drawCapa{(0, 0)}{(2, 0)}{$C_1$}{$Q_1$};
    \end{circuitikz}
  \end{lstlisting}
\end{minipage}

\bigskip

\begin{minipage}{0.2\hsize}
  \begin{circuitikz}
    \drawCapa[l]{(0, 0)}{(2, 0)}{$C_1$}{$Q_1$};
  \end{circuitikz}
\end{minipage}
\begin{minipage}{0.75\hsize}
  \begin{lstlisting}
    \begin{circuitikz}
      \drawCapa[l]{(0, 0)}{(2, 0)}{$C_1$}{$Q_1$};
    \end{circuitikz}
  \end{lstlisting}
\end{minipage}

\bigskip

\begin{minipage}{0.2\hsize}
  \begin{circuitikz}
    \drawCapa[u]{(0, 0)}{(0, 2)}{$C_1$}{$Q_1$};
  \end{circuitikz}
\end{minipage}
\begin{minipage}{0.75\hsize}
  \begin{lstlisting}
    \begin{circuitikz}
      \drawCapa[u]{(0, 0)}{(0, 2)}{$C_1$}{$Q_1$};
    \end{circuitikz}
  \end{lstlisting}
\end{minipage}

\bigskip

\begin{minipage}{0.2\hsize}
  \begin{circuitikz}
    \drawCapa[d]{(0, 0)}{(0, 2)}{$C_1$}{$Q_1$};
  \end{circuitikz}
\end{minipage}
\begin{minipage}{0.75\hsize}
  \begin{lstlisting}
    \begin{circuitikz}
      \drawCapa[d]{(0, 0)}{(0, 2)}{$C_1$}{$Q_1$};
    \end{circuitikz}
  \end{lstlisting}
\end{minipage}

\subsection{コイル}

\begin{minipage}{0.3\hsize}
  \begin{circuitikz}
    \draw (0, 0) to [L] (2, 0) node [right] {\texttt{L}};
  \end{circuitikz}
\end{minipage}
\begin{minipage}{0.65\hsize}
  \begin{lstlisting}
    \begin{circuitikz}
      \draw (0, 0) to [L] (2, 0);
    \end{circuitikz}
  \end{lstlisting}
\end{minipage}

\bigskip

\begin{minipage}{0.3\hsize}
  \begin{circuitikz}
    \draw (0, 0) to [vL] (2, 0) node [right] {\texttt{vL}};
  \end{circuitikz}
\end{minipage}
\begin{minipage}{0.65\hsize}
  \begin{lstlisting}
    \begin{circuitikz}
      \draw (0, 0) to [vL] (2, 0);
    \end{circuitikz}
  \end{lstlisting}
\end{minipage}

\bigskip

\begin{minipage}{0.3\hsize}
  \begin{circuitikz}
    \draw (0, 0) to [cute choke] (2, 0) node [right] {\texttt{cute choke}};
  \end{circuitikz}
\end{minipage}
\begin{minipage}{0.65\hsize}
  \begin{lstlisting}
    \begin{circuitikz}
      \draw (0, 0) to [cute choke] (2, 0);
    \end{circuitikz}
  \end{lstlisting}
\end{minipage}

\bigskip

\begin{minipage}{0.3\hsize}
  \begin{circuitikz}
    \draw (0, 0) to [L, inductors/coils=9] (2, 0);
    \draw (0, -0.5) node [right] {\texttt{L, inductors/coils=9}};
  \end{circuitikz}
\end{minipage}
\begin{minipage}{0.65\hsize}
  \begin{lstlisting}
    \begin{circuitikz}
      \draw (0, 0) to [L, inductors/coils=9] (2, 0);
    \end{circuitikz}
  \end{lstlisting}
\end{minipage}

\bigskip

\begin{minipage}{0.3\hsize}
  \begin{circuitikz}
    \draw (0, 0) to [L, american inductors] (2, 0);
    \draw (0, -0.5) node [right] {\texttt{L, american inductors}} ;
  \end{circuitikz}
\end{minipage}
\begin{minipage}{0.65\hsize}
  \begin{lstlisting}
    \begin{circuitikz}
      \draw (0, 0) to [L, american inductors] (2, 0);
    \end{circuitikz}
  \end{lstlisting}
\end{minipage}

\bigskip

\begin{minipage}{0.3\hsize}
  \begin{circuitikz}
    \draw (0, 0) to [cute choke, twolineschoke] (2, 0) node [right] {\shortstack[l]{\texttt{cute choke,}\\\texttt{twolineschoke}}} ;
  \end{circuitikz}
\end{minipage}
\begin{minipage}{0.65\hsize}
  \begin{lstlisting}
    \begin{circuitikz}
      \draw (0, 0) to [cute choke, twolineschoke] (2, 0);
    \end{circuitikz}
  \end{lstlisting}
\end{minipage}

\subsection{ダイオード}

\begin{minipage}{0.35\hsize}
  \begin{circuitikz}
    \draw (0, 0) to [empty diode] (2, 0) node [right] {\texttt{empty diode}};
  \end{circuitikz}
\end{minipage}
\begin{minipage}{0.6\hsize}
  \begin{lstlisting}
    \begin{circuitikz}
      \draw (0, 0) to [empty diode] (2, 0);
    \end{circuitikz}
  \end{lstlisting}
\end{minipage}

\bigskip

\begin{minipage}{0.35\hsize}
  \begin{circuitikz}
    \draw (0, 0) to [full diode] (2, 0) node [right] {\texttt{full diode}};
  \end{circuitikz}
\end{minipage}
\begin{minipage}{0.6\hsize}
  \begin{lstlisting}
    \begin{circuitikz}
      \draw (0, 0) to [full diode] (2, 0);
    \end{circuitikz}
  \end{lstlisting}
\end{minipage}

\bigskip

\begin{minipage}{0.35\hsize}
  \begin{circuitikz}
    \draw (0, 0) to [stroke diode] (2, 0) node [right] {\texttt{stroke diode}};
  \end{circuitikz}
\end{minipage}
\begin{minipage}{0.6\hsize}
  \begin{lstlisting}
    \begin{circuitikz}
      \draw (0, 0) to [stroke diode] (2, 0);
    \end{circuitikz}
  \end{lstlisting}
\end{minipage}

\bigskip

\begin{minipage}{0.35\hsize}
  \begin{circuitikz}
    \draw (0, 0) to [empty Zener diode] (2, 0) node [right] {\texttt{empty Zener diode}};
  \end{circuitikz}
\end{minipage}
\begin{minipage}{0.6\hsize}
  \begin{lstlisting}
    \begin{circuitikz}
      \draw (0, 0) to [empty Zener diode] (2, 0);
    \end{circuitikz}
  \end{lstlisting}
\end{minipage}

\subsection{電源}

\begin{minipage}{0.35\hsize}
  \begin{circuitikz}
    \draw (0, 0) to [battery] (2, 0) node [right] {\texttt{battery}};
  \end{circuitikz}
\end{minipage}
\begin{minipage}{0.6\hsize}
  \begin{lstlisting}
    \begin{circuitikz}
      \draw (0, 0) to [battery] (2, 0);
    \end{circuitikz}
  \end{lstlisting}
\end{minipage}

\bigskip

\begin{minipage}{0.35\hsize}
  \begin{circuitikz}
    \draw (0, 0) to [battery1] (2, 0) node [right] {\texttt{battery1}};
  \end{circuitikz}
\end{minipage}
\begin{minipage}{0.6\hsize}
  \begin{lstlisting}
    \begin{circuitikz}
      \draw (0, 0) to [battery1] (2, 0);
    \end{circuitikz}
  \end{lstlisting}
\end{minipage}

\bigskip

\begin{minipage}{0.35\hsize}
  \begin{circuitikz}
    \draw (0, 0) to [battery2] (2, 0) node [right] {\texttt{battery2}};
  \end{circuitikz}
\end{minipage}
\begin{minipage}{0.6\hsize}
  \begin{lstlisting}
    \begin{circuitikz}
      \draw (0, 0) to [battery2] (2, 0);
    \end{circuitikz}
  \end{lstlisting}
\end{minipage}

\bigskip

\begin{minipage}{0.35\hsize}
  \begin{circuitikz}
    \draw (0, 0) to [ACV] (2, 0) node [right] {\texttt{ACV}};
  \end{circuitikz}
\end{minipage}
\begin{minipage}{0.6\hsize}
  \begin{lstlisting}
    \begin{circuitikz}
      \draw (0, 0) to [ACV] (2, 0);
    \end{circuitikz}
  \end{lstlisting}
\end{minipage}

\bigskip

\begin{minipage}{0.35\hsize}
  \begin{circuitikz}
    \draw (0, 0) to [sinusoidal voltage source] (2, 0);
    \draw (0, -0.8) node [right] {\texttt{sinusoidal voltage source}};
  \end{circuitikz}
\end{minipage}
\begin{minipage}{0.6\hsize}
  \begin{lstlisting}
    \begin{circuitikz}
      \draw (0, 0) to [sinusoidal voltage source] (2, 0);
    \end{circuitikz}
  \end{lstlisting}
\end{minipage}

\paragraph{Tips}
逆向きに描画したい場合は始点と終点を入れ替えるか,\texttt{invert}を使用する.

\bigskip

\begin{minipage}{0.35\hsize}
  \begin{circuitikz}
    \draw (0, 0) to [battery, invert] (2, 0) node [right] {\texttt{battery, invert}};
  \end{circuitikz}
\end{minipage}
\begin{minipage}{0.6\hsize}
  \begin{lstlisting}
    \begin{circuitikz}
      \draw (0, 0) to [battery, invert] (2, 0);
    \end{circuitikz}
  \end{lstlisting}
\end{minipage}

\subsection{計器}

\begin{minipage}{0.35\hsize}
  \begin{circuitikz}
    \draw (0, 0) to [rmeter, t=A] (2, 0) node [right] {\texttt{rmeter, t=A}};
  \end{circuitikz}
\end{minipage}
\begin{minipage}{0.6\hsize}
  \begin{lstlisting}
    \begin{circuitikz}
      \draw (0, 0) to [rmeter, t=A] (2, 0);
    \end{circuitikz}
  \end{lstlisting}
\end{minipage}

\bigskip

\begin{minipage}{0.35\hsize}
  \begin{circuitikz}
    \draw (0, 0) to [rmeter, t=V] (2, 0) node [right] {\texttt{rmeter, t=V}};
  \end{circuitikz}
\end{minipage}
\begin{minipage}{0.6\hsize}
  \begin{lstlisting}
    \begin{circuitikz}
      \draw (0, 0) to [rmeter, t=V] (2, 0);
    \end{circuitikz}
  \end{lstlisting}
\end{minipage}

\bigskip

\begin{minipage}{0.35\hsize}
  \begin{circuitikz}
    \draw (0, 0) to [ammeter] (2, 0) node [right] {\texttt{ammeter}};
  \end{circuitikz}
\end{minipage}
\begin{minipage}{0.6\hsize}
  \begin{lstlisting}
    \begin{circuitikz}
      \draw (0, 0) to [ammeter] (2, 0);
    \end{circuitikz}
  \end{lstlisting}
\end{minipage}

\bigskip

\begin{minipage}{0.35\hsize}
  \begin{circuitikz}
    \draw (0, 0) to [voltmeter] (2, 0) node [right] {\texttt{voltmeter}};
  \end{circuitikz}
\end{minipage}
\begin{minipage}{0.6\hsize}
  \begin{lstlisting}
    \begin{circuitikz}
      \draw (0, 0) to [voltmeter] (2, 0);
    \end{circuitikz}
  \end{lstlisting}
\end{minipage}

\bigskip

\begin{minipage}{0.35\hsize}
  \begin{circuitikz}
    \draw (0, 0) to [smeter, t=A] (2, 0) node [right] {\texttt{smeter, t=A}};
  \end{circuitikz}
\end{minipage}
\begin{minipage}{0.6\hsize}
  \begin{lstlisting}
    \begin{circuitikz}
      \draw (0, 0) to [smeter, t=A] (2, 0);
    \end{circuitikz}
  \end{lstlisting}
\end{minipage}

\bigskip

\begin{minipage}{0.35\hsize}
  \begin{circuitikz}
    \draw (0, 0) to [smeter, t=V] (2, 0) node [right] {\texttt{smeter, t=V}};
  \end{circuitikz}
\end{minipage}
\begin{minipage}{0.6\hsize}
  \begin{lstlisting}
    \begin{circuitikz}
      \draw (0, 0) to [smeter, t=V] (2, 0);
    \end{circuitikz}
  \end{lstlisting}
\end{minipage}

\bigskip

\begin{minipage}{0.35\hsize}
  \begin{circuitikz}
    \draw (0, 0) to [qiprobe] (2, 0) node [right] {\texttt{qiprobe}};
  \end{circuitikz}
\end{minipage}
\begin{minipage}{0.6\hsize}
  \begin{lstlisting}
    \begin{circuitikz}
      \draw (0, 0) to [qiprobe] (2, 0);
    \end{circuitikz}
  \end{lstlisting}
\end{minipage}

\bigskip

\begin{minipage}{0.35\hsize}
  \begin{circuitikz}
    \draw (0, 0) to [qvprobe] (2, 0) node [right] {\texttt{qvprobe}};
  \end{circuitikz}
\end{minipage}
\begin{minipage}{0.6\hsize}
  \begin{lstlisting}
    \begin{circuitikz}
      \draw (0, 0) to [qvprobe] (2, 0);
    \end{circuitikz}
  \end{lstlisting}
\end{minipage}

\subsection{その他}

\begin{minipage}{0.35\hsize}
  \begin{circuitikz}
    \draw (0, 0) to [lamp] (2, 0) node [right] {\texttt{lamp}};
  \end{circuitikz}
\end{minipage}
\begin{minipage}{0.6\hsize}
  \begin{lstlisting}
    \begin{circuitikz}
      \draw (0, 0) to [lamp] (2, 0);
    \end{circuitikz}
  \end{lstlisting}
\end{minipage}

\bigskip

\begin{minipage}{0.35\hsize}
  \begin{circuitikz}
    \draw (0, 0) to [loudspeaker] (2, 0) node [right] {\texttt{loudspeaker}};
  \end{circuitikz}
\end{minipage}
\begin{minipage}{0.6\hsize}
  \begin{lstlisting}
    \begin{circuitikz}
      \draw (0, 0) to [loudspeaker] (2, 0)
    \end{circuitikz}
  \end{lstlisting}
\end{minipage}

\subsection{交差}

\begin{minipage}{0.35\hsize}
  \begin{circuitikz}
    \draw (-1, 0) to (1, 0);
    \draw (0, 1) to (0, -1);
  \end{circuitikz}
\end{minipage}
\begin{minipage}{0.6\hsize}
  \begin{lstlisting}
    \begin{circuitikz}
      \draw (-1, 0) to (1, 0);
      \draw (0, 1) to (0, -1);
    \end{circuitikz}
  \end{lstlisting}
\end{minipage}

\bigskip

\begin{minipage}{0.35\hsize}
  \begin{circuitikz}
    \draw (-1, 0) to (1, 0);
    \draw (0, 1) to (0, -1);
    \draw (0, 0) node [circ] {};
  \end{circuitikz}
\end{minipage}
\begin{minipage}{0.6\hsize}
  \begin{lstlisting}
    \begin{circuitikz}
      \draw (-1, 0) to (1, 0);
      \draw (0, 1) to (0, -1);
      \draw (0, 0) node [circ] {};
    \end{circuitikz}
  \end{lstlisting}
\end{minipage}

\bigskip

\begin{minipage}{0.35\hsize}
  \begin{circuitikz}
    \draw (-1, 0) to [crossing] (1, 0) node [right] {\texttt{crossing}};
    \draw (0, 1) to (0, -1);
  \end{circuitikz}
\end{minipage}
\begin{minipage}{0.6\hsize}
  \begin{lstlisting}
    \begin{circuitikz}
      \draw (-1, 0) to [crossing] (1, 0);
      \draw (0, 1) to (0, -1);
    \end{circuitikz}
  \end{lstlisting}
\end{minipage}

\subsection{変圧器}
\texttt{node}のオプションにする.

\bigskip

\begin{minipage}{0.3\hsize}
  \begin{circuitikz}
    \draw (0, 0) node [transformer] {};
    \draw (0, -1.5) node {\texttt{transformer}};
  \end{circuitikz}
\end{minipage}
\begin{minipage}{0.65\hsize}
  \begin{lstlisting}
    \begin{circuitikz}
      \draw (0, 0) node [transformer] {};
    \end{circuitikz}
  \end{lstlisting}
\end{minipage}

\bigskip

\begin{minipage}{0.3\hsize}
  \begin{circuitikz}
    \draw (0, 0) node [transformer core] {};
    \draw (0, -1.5) node {\texttt{transformer core}};
  \end{circuitikz}
\end{minipage}
\begin{minipage}{0.65\hsize}
  \begin{lstlisting}
    \begin{circuitikz}
      \draw (0, 0) node [transformer core] {};
    \end{circuitikz}
  \end{lstlisting}
\end{minipage}

\bigskip

\begin{minipage}{0.3\hsize}
  \begin{circuitikz}
    \draw (0, 0) node [transformerNL] {};
    \draw (0, -1.5) node {\texttt{transformerNL}};
  \end{circuitikz}
\end{minipage}
\begin{minipage}{0.65\hsize}
  \begin{lstlisting}
    \begin{circuitikz}
      \draw (0, 0) node [transformerNL] {};
    \end{circuitikz}
  \end{lstlisting}
\end{minipage}

\bigskip

\begin{minipage}{0.3\hsize}
  \begin{circuitikz}
    \draw (0, 0) node [transformerNL core] {};
    \draw (0, -1.5) node {\texttt{transformerNL core}};
  \end{circuitikz}
\end{minipage}
\begin{minipage}{0.65\hsize}
  \begin{lstlisting}
    \begin{circuitikz}
      \draw (0, 0) node [transformerNL core] {};
    \end{circuitikz}
  \end{lstlisting}
\end{minipage}

\begin{minipage}{0.3\hsize}
  \begin{circuitikz}
    \ctikzset{transformer L1/.style={american inductors, inductors/coils=6, inductors/width=0.8}}
    \ctikzset{transformer L2/.style={inductors/coils=10, inductors/width=1.0}}
    \draw (0, 0) node [transformer core] {};
  \end{circuitikz}
\end{minipage}
\begin{minipage}{0.65\hsize}
  \begin{lstlisting}
    \begin{circuitikz}
      \ctikzset{transformer L1/.style={american inductors, inductors/coils=6, inductors/width=0.8}}
      \ctikzset{transformer L2/.style={inductors/coils=10, inductors/width=1.0}}
      \draw (0, 0) node [transformer core] {};
    \end{circuitikz}
  \end{lstlisting}
\end{minipage}

\paragraph{Tips}
変圧器のアンカーは以下の通り.

\begin{center}
  \begin{circuitikz}
    \draw (0, 0) node [transformer, name=trans] {};
    \draw[thick, <-, >=stealth, blue] (trans.inner dot A1) -- +(-1, 0.5) node [above] {inner dot A1};
    \draw[thick, <-, >=stealth, blue] (trans.inner dot A2) -- +(-1, -0.5) node [below] {inner dot A2};
    \draw[thick, <-, >=stealth, blue] (trans.inner dot B1) -- +(1, 0.5) node [above] {inner dot B1};
    \draw[thick, <-, >=stealth, blue] (trans.inner dot B2) -- +(1, -0.5) node [below] {inner dot B2};
    %
    \begin{scope}[shift={(7, 0)}]
      \draw (0, 0) node [transformer, name=trans] {};
      \draw[thick, <-, >=stealth, blue] (trans.outer dot A1) -- +(-1, 0.5) node [above] {outer dot A1};
      \draw[thick, <-, >=stealth, blue] (trans.outer dot A2) -- +(-1, -0.5) node [below] {outer dot A2};
      \draw[thick, <-, >=stealth, blue] (trans.outer dot B1) -- +(1, 0.5) node [above] {outer dot B1};
      \draw[thick, <-, >=stealth, blue] (trans.outer dot B2) -- +(1, -0.5) node [below] {outer dot B2};
    \end{scope}
  \end{circuitikz}
\end{center}

\begin{center}
  \begin{circuitikz}
    \draw (0, 0) node [transformer, name=trans] {};
    \draw[thick, <-, >=stealth, blue] (trans.A1) -- +(0, 0.5) node [above] {A1};
    \draw[thick, <-, >=stealth, blue] (trans.A2) -- +(0, -0.5) node [below] {A2};
    \draw[thick, <-, >=stealth, blue] (trans.B1) -- +(0, 0.5) node [above] {B1};
    \draw[thick, <-, >=stealth, blue] (trans.B2) -- +(0, -0.5) node [below] {B2};
    \draw[thick, <-, >=stealth, blue] (trans.base) -- +(0, 0.5) node [above] {base};
    \draw[thick, <-, >=stealth, blue] (trans-L1.midtap) -- +(1, 0.5) node [right] {L1.midtap};
    \draw[thick, <-, >=stealth, blue] (trans-L2.midtap) -- +(-1, -0.5) node [left] {L2.midtap};
    %
    \begin{scope}[shift={(7, 0)}]
      \draw (0, 0) node [transformer, name=trans] {};
      \draw[thick, <-, >=stealth, blue] (trans-L1.south west) -- +(-0.5, 0.2) node [left] {L1.south west};
      \draw[thick, <-, >=stealth, blue] (trans-L1.south) -- +(-0.5, 0) node [left] {L1.south};
      \draw[thick, <-, >=stealth, blue] (trans-L1.south east) -- +(-0.5, -0.2) node [left] {L1.south east};
      \draw[thick, <-, >=stealth, blue] (trans-L2.south west) -- +(0.5, -0.2) node [right] {L2.south west};
      \draw[thick, <-, >=stealth, blue] (trans-L2.south) -- +(0.5, 0) node [right] {L2.south};
      \draw[thick, <-, >=stealth, blue] (trans-L2.south east) -- +(0.5, 0.2) node [right] {L2.south east};
    \end{scope}
  \end{circuitikz}
\end{center}

\begin{center}
  \begin{circuitikz}
    \draw (0, 0) node [transformer, name=trans] {};
    \draw[thick, <-, >=stealth, blue] (trans-L1.north west) -- +(0, 1) node [left] {L1.north west};
    \draw[thick, <-, >=stealth, blue] (trans-L1.north) -- +(1, 0.5) node [right] {L1.north};
    \draw[thick, <-, >=stealth, blue] (trans-L1.north east) -- +(0, -1) node [left] {L1.north east};
    \draw[thick, <-, >=stealth, blue] (trans-L2.north west) -- +(0, -1) node [right] {L2.north west};
    \draw[thick, <-, >=stealth, blue] (trans-L2.north) -- +(-1, -0.5) node [left] {L2.north};
    \draw[thick, <-, >=stealth, blue] (trans-L2.north east) -- +(0, 1) node [right] {L2.north east};

    \begin{scope}[shift={(7, 0)}]
      \draw (0, 0) node [transformer, name=trans] {};
      \draw[thick, <-, >=stealth, blue] (trans-L1.east) -- +(-0.5, 0.2) node [left] {L1.east};
      \draw[thick, <-, >=stealth, blue] (trans-L1.west) -- +(-0.5, -0.2) node [left] {L1.west};
      \draw[thick, <-, >=stealth, blue] (trans-L2.east) -- +(0.5, -0.2) node [right] {L2.east};
      \draw[thick, <-, >=stealth, blue] (trans-L2.west) -- +(0.5, 0.2) node [right] {L2.west};
      \draw[thick, <-, >=stealth, blue] (trans-L1.a) -- +(-0.5, -0.2) node [left] {L1.a};
      \draw[thick, <-, >=stealth, blue] (trans-L1.b) -- +(-0.5, 0.2) node [left] {L1.b};
      \draw[thick, <-, >=stealth, blue] (trans-L2.a) -- +(0.5, 0.2) node [right] {L2.a};
      \draw[thick, <-, >=stealth, blue] (trans-L2.b) -- +(0.5, -0.2) node [right] {L2.b};
    \end{scope}
  \end{circuitikz}
\end{center}

\subsection{スイッチ}

\begin{minipage}{0.35\hsize}
  \begin{circuitikz}
    \draw (0, 0) to [opening switch] (2, 0) node [right] {\texttt{opening switch}};
  \end{circuitikz}
\end{minipage}
\begin{minipage}{0.6\hsize}
  \begin{lstlisting}
    \begin{circuitikz}
      \draw (0, 0) to [opening switch] (2, 0);
    \end{circuitikz}
  \end{lstlisting}
\end{minipage}

\bigskip

\begin{minipage}{0.35\hsize}
  \begin{circuitikz}
    \draw (0, 0) to [closing switch] (2, 0) node [right] {\texttt{closing switch}};
  \end{circuitikz}
\end{minipage}
\begin{minipage}{0.6\hsize}
  \begin{lstlisting}
    \begin{circuitikz}
      \draw (0, 0) to [closing switch] (2, 0);
    \end{circuitikz}
  \end{lstlisting}
\end{minipage}

\bigskip

\begin{minipage}{0.35\hsize}
  \begin{circuitikz}
    \draw (0, 0) to [cute open switch] (2, 0) node [right] {\texttt{cute open switch}};
  \end{circuitikz}
\end{minipage}
\begin{minipage}{0.6\hsize}
  \begin{lstlisting}
    \begin{circuitikz}
      \draw (0, 0) to [cute open switch] (2, 0);
    \end{circuitikz}
  \end{lstlisting}
\end{minipage}

\bigskip

\begin{minipage}{0.35\hsize}
  \begin{circuitikz}
    \draw (0, 0) to [cute closed switch] (2, 0) node [right] {\texttt{cute closed switch}};
  \end{circuitikz}
\end{minipage}
\begin{minipage}{0.6\hsize}
  \begin{lstlisting}
    \begin{circuitikz}
      \draw (0, 0) to [cute closed switch] (2, 0);
    \end{circuitikz}
  \end{lstlisting}
\end{minipage}

\bigskip

\begin{minipage}{0.35\hsize}
  \begin{circuitikz}
    \draw (0, 0) to [cute opening switch] (2, 0);
    \draw (0, -0.5) node [right] {\texttt{cute opening switch}};
  \end{circuitikz}
\end{minipage}
\begin{minipage}{0.6\hsize}
  \begin{lstlisting}
    \begin{circuitikz}
      \draw (0, 0) to [cute opening switch] (2, 0);
    \end{circuitikz}
  \end{lstlisting}
\end{minipage}

\bigskip

\begin{minipage}{0.35\hsize}
  \begin{circuitikz}
    \draw (0, 0) to [cute closing switch] (2, 0);
    \draw (0, -0.5) node [right] {\texttt{cute closing switch}};
  \end{circuitikz}
\end{minipage}
\begin{minipage}{0.6\hsize}
  \begin{lstlisting}
    \begin{circuitikz}
      \draw (0, 0) to [cute closing switch] (2, 0);
    \end{circuitikz}
  \end{lstlisting}
\end{minipage}

\bigskip

\begin{minipage}{0.35\hsize}
  \begin{circuitikz}
    \draw (0, 0) node [cute spdt up, name=Sw] {};
    \draw[thick, <-, >=stealth, blue] (Sw.in) -- +(0, 0.3) node [above] {in};
    \draw[thick, <-, >=stealth, blue] (Sw.out 1) -- +(0.5, 0) node [right] {out 1};
    \draw[thick, <-, >=stealth, blue] (Sw.out 2) -- +(0.5, 0) node [right] {out 2};
    \draw (2, 0) node [right] {\texttt{cute spdt up}};
  \end{circuitikz}
\end{minipage}
\begin{minipage}{0.6\hsize}
  \begin{lstlisting}
    \begin{circuitikz}
      \draw (0, 0) node [cute spdt up] {};
    \end{circuitikz}
  \end{lstlisting}
\end{minipage}

\bigskip

\begin{minipage}{0.35\hsize}
  \begin{circuitikz}
    \draw (0, 0) node [cute spdt mid, name=Sw] {};
    \draw[thick, <-, >=stealth, blue] (Sw.in) -- +(0, 0.3) node [above] {in};
    \draw[thick, <-, >=stealth, blue] (Sw.out 1) -- +(0.5, 0) node [right] {out 1};
    \draw[thick, <-, >=stealth, blue] (Sw.out 2) -- +(0.5, 0) node [right] {out 2};
    \draw (2, 0) node [right] {\texttt{cute spdt mid}};
  \end{circuitikz}
\end{minipage}
\begin{minipage}{0.6\hsize}
  \begin{lstlisting}
    \begin{circuitikz}
      \draw (0, 0) node [cute spdt mid] {};
    \end{circuitikz}
  \end{lstlisting}
\end{minipage}

\bigskip

\begin{minipage}{0.35\hsize}
  \begin{circuitikz}
    \draw (0, 0) node [cute spdt down, name=Sw] {};
    \draw[thick, <-, >=stealth, blue] (Sw.in) -- +(0, 0.3) node [above] {in};
    \draw[thick, <-, >=stealth, blue] (Sw.out 1) -- +(0.5, 0) node [right] {out 1};
    \draw[thick, <-, >=stealth, blue] (Sw.out 2) -- +(0.5, 0) node [right] {out 2};
    \draw (2, 0) node [right] {\texttt{cute spdt down}};
  \end{circuitikz}
\end{minipage}
\begin{minipage}{0.6\hsize}
  \begin{lstlisting}
    \begin{circuitikz}
      \draw (0, 0) node [cute spdt down] {};
    \end{circuitikz}
  \end{lstlisting}
\end{minipage}

\bigskip

\begin{minipage}{0.35\hsize}
  \begin{circuitikz}
    \draw (0, 0) node [cute spdt up arrow, name=Sw] {};
    \draw[thick, <-, >=stealth, blue] (Sw.in) -- +(0, 0.3) node [above] {in};
    \draw[thick, <-, >=stealth, blue] (Sw.out 1) -- +(0.5, 0) node [right] {out 1};
    \draw[thick, <-, >=stealth, blue] (Sw.out 2) -- +(0.5, 0) node [right] {out 2};
    \draw (1, 0) node [right] {\texttt{cute spdt up arrow}};
  \end{circuitikz}
\end{minipage}
\begin{minipage}{0.6\hsize}
  \begin{lstlisting}
    \begin{circuitikz}
      \draw (0, 0) node [cute spdt up arrow] {};
    \end{circuitikz}
  \end{lstlisting}
\end{minipage}

\bigskip

\begin{minipage}{0.35\hsize}
  \begin{circuitikz}
    \draw (0, 0) node [cute spdt mid arrow, name=Sw] {};
    \draw[thick, <-, >=stealth, blue] (Sw.in) -- +(0, 0.3) node [above] {in};
    \draw[thick, <-, >=stealth, blue] (Sw.out 1) -- +(0.5, 0) node [right] {out 1};
    \draw[thick, <-, >=stealth, blue] (Sw.out 2) -- +(0.5, 0) node [right] {out 2};
    \draw (1, 0) node [right] {\texttt{cute spdt mid arrow}};
  \end{circuitikz}
\end{minipage}
\begin{minipage}{0.6\hsize}
  \begin{lstlisting}
    \begin{circuitikz}
      \draw (0, 0) node [cute spdt mid arrow] {};
    \end{circuitikz}
  \end{lstlisting}
\end{minipage}

\bigskip

\begin{minipage}{0.35\hsize}
  \begin{circuitikz}
    \draw (0, 0) node [cute spdt down arrow, name=Sw] {};
    \draw[thick, <-, >=stealth, blue] (Sw.in) -- +(0, 0.3) node [above] {in};
    \draw[thick, <-, >=stealth, blue] (Sw.out 1) -- +(0.5, 0) node [right] {out 1};
    \draw[thick, <-, >=stealth, blue] (Sw.out 2) -- +(0.5, 0) node [right] {out 2};
    \draw (1, 0) node [right] {\texttt{cute spdt down arrow}};
  \end{circuitikz}
\end{minipage}
\begin{minipage}{0.6\hsize}
  \begin{lstlisting}
    \begin{circuitikz}
      \draw (0, 0) node [cute spdt down arrow] {};
    \end{circuitikz}
  \end{lstlisting}
\end{minipage}

\section{サンプル}
\subsection{トランス回路}
\begin{circuitikz}
  \ctikzset{transformer L1/.style={inductors/coils=6, inductors/width=1.0}}
  \ctikzset{transformer L2/.style={inductors/coils=10, inductors/width=1.0}}
  \draw (0, 0) node [transformerNL core, name=trans] {};
  \coordinate (UL) at ($ (trans.A1) + (-3, 0) $);
  \coordinate (DL) at ($ (trans.A2) + (-3, 0) $);
  \coordinate (UR) at ($ (trans.B1) + (3, 0) $);
  \coordinate (DR) at ($ (trans.B2) + (3, 0) $);
  \draw (trans.A1) to [R, l^=$R_1$, f_<=$i_1$, *-] (UL) to [sinusoidal voltage source, l_=$V_0\sin\omega t$] (DL) to [short, -*] (trans.A2);
  \draw (trans.B1) to [short, f>^=$i_2$, *-] (UR) to [R, l=$R_2$] (DR) to [cute closed switch, -*] (trans.B2);
  \draw (trans.A1) node [above] {a};
  \draw (trans.A2) node [below] {b};
  \draw (trans.B1) node [above] {c};
  \draw (trans.B2) node [below] {d};
\end{circuitikz}

\bigskip

\begin{lstlisting}
  \begin{circuitikz}
    \ctikzset{transformer L1/.style={inductors/coils=6, inductors/width=1.0}}
    \ctikzset{transformer L2/.style={inductors/coils=10, inductors/width=1.0}}
    \draw (0, 0) node [transformerNL core, name=trans] {};
    \coordinate (UL) at ($ (trans.A1) + (-3, 0) $);
    \coordinate (DL) at ($ (trans.A2) + (-3, 0) $);
    \coordinate (UR) at ($ (trans.B1) + (3, 0) $);
    \coordinate (DR) at ($ (trans.B2) + (3, 0) $);
    \draw (trans.A1) to [R, l^=$R_1$, f_<=$i_1$, *-] (UL) to [sinusoidal voltage source, l_=$V_0\sin\omega t$] (DL) to [short, -*] (trans.A2);
    \draw (trans.B1) to [short, f>^=$i_2$, *-] (UR) to [R, l=$R_2$] (DR) to [cute closed switch, -*] (trans.B2);
    \draw (trans.A1) node [above] {a};
    \draw (trans.A2) node [below] {b};
    \draw (trans.B1) node [above] {c};
    \draw (trans.B2) node [below] {d};
  \end{circuitikz}
\end{lstlisting}

\newpage

\subsection{コンデンサー回路1}

\begin{circuitikz}
  \draw (0, 0) to [battery1, l=$V$] (0, -2) to [short] (4, -2);
  \draw (0, 0) to [cute open switch=S$_1$] (2,0) to [cute closed switch=S$_2$] (4,0);
  \drawCapa[u]{(2, -2)}{(2, 0)}{$C_1$}{$Q_1$}
  \drawCapa[d]{(4, -2)}{(4, 0)}{$C_2$}{$Q_2$}
\end{circuitikz}

\bigskip

\begin{lstlisting}
  \begin{circuitikz}
    \draw (0, 0) to [battery1, l=$V$] (0, -2) to [short] (4, -2);
    \draw (0, 0) to [cute open switch=S$_1$] (2,0) to [cute closed switch=S$_2$] (4,0);
    \drawCapa[u]{(2, -2)}{(2, 0)}{$C_1$}{$Q_1$}
    \drawCapa[d]{(4, -2)}{(4, 0)}{$C_2$}{$Q_2$}
  \end{circuitikz}
\end{lstlisting}

% \newpage

\subsection{コンデンサー回路2}
\begin{circuitikz}
  \draw (0, 0) node [ground] {} node [left] {G} to [battery2, name=battery, invert, *-] (0, 4) to [cute open switch, l=S$_1$, -*] (2, 4) to [cute open switch, l=S$_2$] (4, 4) to [R, l=R, -*] (6, 4) node [above right] {A} to [capacitor, l=C$_1$, -*] (6, 2) to [capacitor, l=C$_2$] (6, 0) to [short, -*] (2, 0) to (0, 0);
  \draw (2, 0) to [short, -*] (2, 2) to [capacitor, l^=C$_0$] (2, 4);
  \draw (2, 2) to [cute open switch, l_=S$_3$] (6, 2);
  \draw (battery.north east) node {E};
\end{circuitikz}

\begin{lstlisting}
  \begin{circuitikz}
    \draw (0, 0) node [ground] {} node [left] {G} to [battery2, name=battery, invert, *-] (0, 4) to [cute open switch, l=S$_1$, -*] (2, 4) to [cute open switch, l=S$_2$] (4, 4) to [R, l=R, -*] (6, 4) node [above right] {A} to [capacitor, l=C$_1$, -*] (6, 2) to [capacitor, l=C$_2$] (6, 0) to [short, -*] (2, 0) to (0, 0);
    \draw (2, 0) to [short, -*] (2, 2) to [capacitor, l^=C$_0$] (2, 4);
    \draw (2, 2) to [cute open switch, l_=S$_3$] (6, 2);
    \draw (battery.north east) node {E};
  \end{circuitikz}
\end{lstlisting}

\newpage

\subsection{ダイオード回路}
\begin{circuitikz}[american voltages]
  \draw (0, 0) node [ground] {} to [sinusoidal voltage source, v^<={\null}] (0, 3) to (1.5, 3);
  \draw (1.5, 3) to [stroke diode, invert, xf=\relax] (3, 4.5) to [stroke diode] (4.5, 3) to [stroke diode] (3, 1.5) to [stroke diode, invert] (1.5, 3);
  \draw (4.5, 3) to (5, 3) to (5, 0) to (0, 0);
  \draw (3, 1.5) to (4.5, 1.5) to [crossing] (5.5, 1.5) node [right] {b} to [R, l_=R] (5.5, 4.5) node [right] {a} to (3, 4.5);
\end{circuitikz}

\begin{lstlisting}
  \begin{circuitikz}[american voltages]
    \draw (0, 0) node [ground] {} to [sinusoidal voltage source, v^<={\null}] (0, 3) to (1.5, 3);
    \draw (1.5, 3) to [stroke diode, invert, xf=\relax] (3, 4.5) to [stroke diode] (4.5, 3) to [stroke diode] (3, 1.5) to [stroke diode, invert] (1.5, 3);
    \draw (4.5, 3) to (5, 3) to (5, 0) to (0, 0);
    \draw (3, 1.5) to (4.5, 1.5) to [crossing] (5.5, 1.5) node [right] {b} to [R, l_=R] (5.5, 4.5) node [right] {a} to (3, 4.5);
  \end{circuitikz}
\end{lstlisting}

\newpage

\section{参考}
素子等は以下のファイルで定義される.
新しい素子を作る際の参考に.
kpsewhichで開ける.

\begin{itemize}
  \item ctikzstyle-example.tex
  \item pgfcircflow.tex
  \item pgfcircshapes.tex
  \item ctikzstyle-legacy.tex
  \item pgfcirclabel.tex
  \item pgfcirctripoles.tex
  \item ctikzstyle-romano.tex
  \item pgfcircmonopoles.tex
  \item pgfcircutils.tex
  \item pgfcirc.defines.tex
  \item pgfcircmultipoles.tex
  \item pgfcircvoltage.tex
  \item pgfcircbipoles.tex
  \item pgfcircpath.tex
  \item pgfcirccurrent.tex
  \item pgfcircquadpoles.tex
\end{itemize}

\end{document}
