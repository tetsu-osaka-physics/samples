\documentclass[a4paper, papersize, dvipdfmx, fleqn, f, nobox, anslabel, bold, ideographic]{jsarticle}
\usepackage{exam_s}
\begin{document}
\section{}\fstyle{Kakko}

\begin{center}
  \setlength{\arrayrulewidth}{1.2pt}
  \begin{tabular}{p{0.31\hsize}p{0.31\hsize}p{0.31\hsize}}\hline \\[-.6zw]
    & & \hfil [A]各1点,計10点 \\
    \ans & \multicolumn{2}{l}{\ans} \\
    \ans & \ans & \ans \\
    \ans & \ans & \ans \\
    \ans & \ans & \\
    & & \hfil [B]各2点,計10点 \\
    \ans & \ansskip & \ans \\
    \ans & \ans & \\
    \hline
  \end{tabular}
  \setcounter{Fcntr}{0}
\end{center}

\subsection{}
A, Bは\verb|\subsection{}|で出す。

\verb|\uw|で波線引くと,\verb|\fstyle|で指定した形式で番号が自動的に入る:
\[\uw{A}\]

第1問以外はenumerate環境つかわなくてもいい。
3分の1のスペースで収まらない場合は,\f*{2}みたいにすればいい。

図とかを答えさせる場合は,表の\verb|\ans|は\verb|\ansskip|にする。

\uw{とてもとても長い解答とてもとても長い解答}
\uw{3}\uw{4}\uw{5}\uw{6}\uw{7}\uw{8}\uw{9}\uw{10}

\subsection{}
\uw{11}

12番目の解答は略するので,カウンターを一つ進める:\verb|\addtocounter{Fcntr}{1}|
\addtocounter{Fcntr}{1}

\uw{13}\uw{14}\uw{\protect\ajMaru{15}}

\verb|\ajMaru|とかを解答にする場合は\verb|\protect|をつける。

\end{document}
